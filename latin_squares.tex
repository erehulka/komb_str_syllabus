\chapter{Latinské štvorce}
Latinský štvorec, s ktorým budeme túto kapitolu pracovať predstavuje štvorcovú tabuľku rozmerov $n \times n$ vyplnenú $n$ rôznymi symbolmi, napríklad číslami od $1$ po $n$ tak, že v každom riadku a v každom stĺpci sa každý symbol vyskytuje práve raz. Latinské štvorce nachádzajú uplatnenie vo viacerých oblastiach ako štatistika, lineárna algebra, kódovanie alebo aj pri rôznych logických úlohách a hlavolamoch. 

Jednoduchou ilustráciou latinských štvorcov je napríklad Sudoku, ktoré je vlastne zosilnený latinský štvorec o požiadavky na štvorcové bloky $3 \times 3$.

\section{Definícia, základné vlastnosti}
Okrem formálnej definície latinského štvorca, pripomenieme definíciu permutácie a jej základných vlastností. Riadky a stĺpce latinského štvorca si vieme predstaviť ako permutácie prvkov $1, \ldots, n$, čo neskôr využijeme na alternatívnu definíciu latinského štvorca.

\begin{definition}
Tabuľka rozmerov $n\times n$ s prvkami z $\set{1, \ldots, n}$ je latinský~štvorec rádu $n$, ak platí:
\begin{enumerate}
    \item v každom riadku sa vyskytuje všetkých $n$ rôznych symbolov
    \item v každom stĺpci sa vyskytuje všetkých $n$ rôznych symbolov
\end{enumerate}
\end{definition}

\begin{exercise}
Zostrojte latinský štvorec rádu 5.
\end{exercise}

\begin{exercise}
Zostrojte latinský štvorec rádu 7.
\end{exercise}

\begin{exercise}
Napíšte program na generovanie latinských štvorcov, ktorý bude generovať štvorec po riadkoch preberaním všetkých možností.
\end{exercise}

\begin{example}
Zostrojovať a generovať latinské štvorce je príjemná úloha na precvičenie definície, no môžeme sa aj zamyslieť načo by nám mohli byť užitočné. Predstavme si nasledujúci problém: \textit{Sme vlastník farmy a chceme rozšíriť produkciu o vajíčka. Máme na výber z troch druhov sliepok $s_1, s_2$ a $s_3$. Chceme samozrejme vybrať tie, čo znesú najviac vajec za nejaké obdobie, napríklad 3 mesiace - apríl, máj a jún}.

Na takéto porovnanie môžeme navrhnúť jednoduchý experiment:
\begin{itemize}
    \item Kúpime $3$ polia $p_1, p_2, p_3$ pre pasenie sliepok.
    \item Umiestnime druh $s_1$ na pole $p_1$, druh $s_2$ na pole $p_2$ a druh $s_3$ na pole $p_3$.
    \item Spočítame počet znesených vajec za tri mesiace pre jednotlivé druhy sliepok a vyberieme druh, ktorý zniesol najviac.
\end{itemize}

Tento nápad má však zjavný problém, jeden druh testujeme iba na jednom poli. Teda produkcia vajec daného druhu nemusí záležať len od vlastností daného druhu, ale aj od poľa, na ktorom sa pásli. Napríklad pole $p_1$ je vystavené väčšiemu hluku, pole $p_2$ je bližšie k lesu, kde žije veľa líšok a pole $p_3$ je podmývané.

Lepším riešením by bolo testovať všetky druhy na všetkých poliach, na každom aspoň mesiac. Aby sme nemuseli kupovať ďalšie polia alebo nerealizovali experiment dlhšie, navrhneme rozpis, na ktorom poli bude, v ktorom mesiaci, ktorý druh sliepok. Tento rozpis môžeme reprezentovať a zostrojiť ako latinský štvorec, kde riadky budú polia, stĺpce mesiace a v bunkách budú druhy sliepok.

\end{example}

\begin{definition}
Dvojica $S_n = (P, \circ)$ je grupa permutácií rádu $n$, kde $P$ je množina všetkých permutácii veľkosti $n$ (t.j. množina všetkých bijektívnych zobrazení z množiny $\set{1, \ldots, n}$ na ňu samu) a $\circ$ je binárna operácia skladania zobrazení, definovaná nasledovné: $$\forall\phi,\psi\in S_n  \forall x \in \set{1, \ldots, n}: (\phi \circ \psi)(x) = \phi(\psi(x)).$$ 
Pre krátkosť zápisu budeme symbol skladania vynechávať, t.j. $\phi \psi \overset{def}{\equiv} \phi \circ \psi$. 
Neutrálny prvok grupy $S_n$ (t.j. identické zobrazenie) budeme označovať $1$.
\end{definition}

\begin{definition}
\label{def:permdist}
Nech $\phi, \psi \in S_n$ sú permutácie veľkosti $n$. 
Potom vzdialenosť $\dist(\phi, \psi)$ dvoch permutácií definujeme ako počet prvkov, ktoré dané permutácie zobrazia rôzne. 
Formálne, 
$$\dist(\phi, \psi) := \left| \set{x ~|~ x \in \set{1, \ldots, n} \wedge \phi(x) \neq \psi(x) } \right| $$
\end{definition}

\begin{definition}
Nech $\phi \in S_n$ je permutácia veľkosti $n$. 
Potom $Fix(\phi)$ je množina všetkých pevných bodov permutácie $\phi$. 
Formálne, 
$$Fix(\phi) := \set{x ~|~ x \in \set{1, \ldots, n} \wedge \phi(x) = x}$$ 
\end{definition}


\begin{exercise}
Dokážte, že platí $\forall \phi \in S_n: Fix(\phi) = Fix(\phi^{-1})$.
\end{exercise}



\begin{theorem}
\label{thm:permdist}
Nech $\phi, \psi \in S_n$ sú permutácie veľkosti $n$. Potom platí:

\begin{enumerate}
    \item $\forall \lambda \in S_n: \dist(\phi \lambda, \psi \lambda) = \dist(\lambda \phi, \lambda \psi) = \dist(\phi, \psi)$
    \item $\dist(\phi, \psi) = \dist(1, \phi^{-1} \psi) = n - |Fix(\phi^{-1} \psi)|$
\end{enumerate}
\end{theorem}
\begin{proof}
Začneme dôkazom prvého tvrdenia. 
Dokážeme rovnosť $\dist(\phi \lambda, \psi \lambda) = \dist(\phi, \psi)$, druhú rovnosť prenechávame čitateľovi ako samostatné cvičenie (úloha \ref{exer:permdist1}).

Chceme ukázať, že vzdialenosti dvoch párov permutácií sú rovnaké. 
To znamená, že počty prvkov, v ktorých sa ich hodnoty líšia, je rovnaký.
Prvú množinu prvkov označíme ako $A$, druhú ako $B$. 
Formálne, nech 
$$A := \set{x | x \in \set{1, \ldots, n} \wedge \phi(x) \neq \psi(x)}$$
(t.j. $\dist(\phi, \psi) = |A|$ z definície vzdialenosti \ref{def:permdist}) a 
$$B := \set{y | y \in \set{1, \ldots, n} \wedge \phi(\lambda(y)) \neq \psi(\lambda(y))}$$ 
(t.j. $\dist(\phi\lambda, \psi\lambda) = |B|$ z definície vzdialenosti \ref{def:permdist}).

Najprv ukážeme, že platí $|A| \geq |B|$, následne $|B| \geq |A|$.
Z toho už platnosť prvého tvrdenia z vety bude očividná.

Nech $x \in A$. 
Z definície množiny $A$ vyplýva, že $\phi(x) \neq \psi(x)$. 
Nech $y := \lambda^{-1}(x)$. Potom platí, že $\phi(\lambda(y)) \neq \psi(\lambda(y))$, čiže $y \in B$ z definície množiny $B$.
Keďže $\lambda^{-1}$ je permutácia, tak je injektívnym zobrazením. 
To znamená, že rôzne hodnoty $x$ premietne do rôznych hodnôt $y$. 
Z toho vyplýva, že veľkosť množiny $B$ je aspoň tak veľká, ako množiny $A$, t.j. $|A| \leq |B|$.

Ukážeme teraz druhú nerovnosť. 
Nech $y \in B$. 
Potom (z definície množiny $B$) platí $\phi(\lambda(y)) \neq \psi(\lambda(y))$. 
Teda, z definície množiny $A$, $\lambda(y) \in A$.
Analogicky (z injektivity permutácie $\lambda$) vyplýva, že $|B| \leq |A|$.
Týmto je dôkaz prvého tvrdenia z~dokazovanej vety ukončený.

Dôkaz druhého tvrdenia z vety prenechávame čitateľovi ako samostatné cvičenie (úloha \ref{exer:permdist2}).
\end{proof}

\begin{exercise}
\label{exer:permdist1}
Dokážte, že $\forall \phi, \psi, \lambda \in S_n: \dist(\lambda\phi, \lambda\psi) = \dist(\phi, \psi)$.
\end{exercise}

\begin{exercise}
\label{exer:permdist2}
Dokážte tvrdenie 2 z vety \ref{thm:permdist} (\emph{hint: použite prvé tvrdenie z tejto vety}).
\end{exercise}


\begin{exercise}
Dokážte, že $\forall \phi, \psi \in S_n: \dist(\phi, \psi) = \dist(\phi^{-1}, \psi^{-1})$.
\end{exercise}



\begin{theorem}
    Funkcia $\dist(\phi, \psi)$ je metrikou priestoru $S_n$, t.j. ona spĺňa nasledujúce podmienky:
    \begin{enumerate}
        \item $ \forall \phi, \psi \in S_n: \dist(\phi, \psi) = 0 \Leftrightarrow \phi = \psi$
        \item $ \forall \phi, \psi \in S_n: \dist(\phi, \psi) = \dist(\psi, \phi)$ (symetria)
        \item $ \forall \phi, \psi, \lambda \in S_n: \dist(\phi, \psi) + \dist(\psi, \lambda) \geq \dist(\phi, \lambda)$ (trojuholníková nerovnosť)  
    \end{enumerate}
\end{theorem}

\begin{proof}
Prvé dve tvrdenia sú očividné a nepotrebujú špeciálny dôkaz.
Pozrime sa teda na tretie tvrdenie o trojuholníkovej nerovnosti.

Nech $\phi_1, \phi_2, \phi_3 \in S_n$. 
Treba dokázať, že\footnote{tu a ďalej: otázniky nad znakmi rovnosti, nerovnosti alebo iných relácií označujú ten fakt, že dané tvrdenia ešte nie sú dokázané v rámci textu. 
Používame túto notáciu najmä v prípadoch, keď robíme úpravy nad práve dokazovanými tvrdeniami} 
$$\dist(\phi_1, \phi_2) + \dist(\phi_2, \phi_3) \overset{?}{\geq} \dist(\phi_1, \phi_3).$$
Z vety \ref{thm:permdist} táto nerovnosť je ekvivalentná s nerovnosťou\footnote{väčšina učebnicových dôkazov má veľmi jednoduchú schému: skúšame použiť postupne všetky doteraz dokázané vety. 
V danom prípade prehľadávanie je uľahčené tým faktom, že toto je ešte len druhá veta}
$$(n - |Fix(\phi_1^{-1}\phi_2)|) + (n - |Fix(\phi_2^{-1}\phi_3)|) \overset{?}{\geq} n - |Fix(\phi_1^{-1}\phi_3)|.$$
Ak komplement množiny $A$ (vzhľadom na základnú množinu $\set{1, \ldots, n}$) označíme ako $\overline{A}$, tak túto nerovnosť môžeme prepísať do ďalšieho ekvivalentného tvaru\footnote{pozor, v tretej kapitole si podobným symbolom budeme označovať uzáver množiny v matroide}:
$$|\overline{Fix(\phi_1^{-1}\phi_2)}| + |\overline{Fix(\phi_2^{-1}\phi_3)}| \overset{?}{\geq} |\overline{Fix(\phi_1^{-1}\phi_3)}|.$$

Dokážeme si pomocné tvrdenie\footnote{niektorí ľudia v týchto okamihoch zvyknú nahlas povedať "Heuréka!"}: 
$$\overline{Fix(\phi_1^{-1}\phi_2)} \cup \overline{Fix(\phi_2^{-1}\phi_3)} \overset{?}{\supseteq} \overline{Fix(\phi_1^{-1}\phi_3)}.$$

Všimnite si, že ak daná inklúzia platí, tak z nej vyplývajú aj predchádzajúce tvrdenia (kvôli $A \cup B \supseteq C \Longrightarrow |A \cup B| \geq |C| \Longrightarrow |A| + |B| \geq |C|$).
Môžeme toto tvrdenie prepísať do ekvivalentného tvaru, odstrániac znaky komplementu:
$$Fix(\phi_1^{-1}\phi_2) \cap Fix(\phi_2^{-1}\phi_3) \overset{?}{\subseteq} Fix(\phi_1^{-1}\phi_3).$$

Túto inklúziu dokážeme priamo. 
Nech $x \in Fix(\phi_1^{-1}\phi_2) \cap Fix(\phi_2^{-1}\phi_3)$. 
Potom $x = \phi_1^{-1}(\phi_2(x))$ a $x = \phi_2^{-1}(\phi_3(x))$.
Dosadením druhej rovnosti do prvej dostaneme $x = \phi_1^{-1}(\phi_2(\phi_2^{-1}(\phi_3(x)))) = \phi_1^{-1}(\phi_3(x))$, čiže $x$ je aj pevným bodom permutácie $\phi_1^{-1} \phi_3$, t.j. $x \in Fix(\phi_1^{-1}\phi_3)$.
Týmto je dôkaz pomocného tvrdenia, a s tým aj celej vety, ukončený.
\end{proof}

\begin{definition}
Latinský obdĺžnik rozmerov $k \times n$ je postupnosť $L = [\phi_1, \phi_2, \ldots, \phi_k]$ permutácií z $S_n$ takých, že všetky sú vo vzdialenosti $n$. 
Formálne, 
$$\forall i, j \in \set{1, \ldots, n}: i \neq j \Rightarrow \dist(\phi_i, \phi_j) = n$$
\end{definition}

\begin{definition}{(Iná definícia latinských štvorcov)}
Latinský štvorec rádu $n$ je latinský obdĺžnik typu $k \times n$ s maximálnou dĺžkou postupnosti. 
Inak povedané, latinský štvorec rádu $n$ je postupnosť $n$ permutácií z $S_n$, ktoré sú od seba vzdialené $n$.
\end{definition}


\begin{exercise}
Nech $l = [\phi_1, \ldots, \phi_k]$ je latinský obdĺžnik typu $k \times n$ a $\lambda \in S_n$ je permutácia rádu $n$.
Dokážte, že $[\lambda \phi_1, \ldots, \lambda \phi_k]$ a $[\phi_1 \lambda, \ldots, \phi_k \lambda]$ sú tiež latinskými obdĺžnikmi typu $k \times n$.
\end{exercise}


\begin{definition}
Nech X je množina a $\mathcal{X} = \set{X_1, \ldots, X_k}, \forall i \in \set{1, \ldots, k}: X_i \subseteq X$ je systém jej podmnožín.
Systém rozličných reprezentantov pre $\mathcal{X}$ je postupnosť $x_1, \ldots x_k, \forall i \in \set{1, \ldots, k}: x_i \in X_i$ 
a zároveň sú všetky jej prvky rôzne, teda $\forall i, j \in \set{1, \ldots, k}: i \neq j \Rightarrow  x_i \neq x_j$.
\end{definition}

\begin{lemma}{(Hallova veta o párení, 1935)}
Nech $G = (A \cup B, E)$, kde $E \subseteq A \times B$, je konečný bipartitný graf s partíciami $A$ a $B$.
Nech $N_G(W)$ je okolie množiny vrcholov $W \subseteq A$.
Formálne, $N_G(W) := \set{y \in B| \exists x \in W: (x, y) \in E} $
Potom graf $G$ má úplné párenie práve vtedy keď $$\forall W \subseteq A: |N_G(W)| \geq |W|.$$
Neformálne povedané, ak každá podmnožina vrcholov z $A$ má dostatočný počet kandidátov na spárovanie.
\end{lemma}
Dôkaz tejto lemy môžete nájsť v knihe \emph{Graph Theory} (Diestel, 2000)\footnote{alebo na Wikipédii: \href{https://en.wikipedia.org/wiki/Hall\%27s_marriage_theorem}{https://en.wikipedia.org/wiki/Hall's\_marriage\_theorem}}.

\begin{theorem}{(Hallova veta pre množiny)}
    Nech $\mathcal{X}$ je systém podmnožín množiny $X$. 
    Ak $\forall \mathcal{Y} = \set{Y_1, \ldots, Y_m} \subseteq \mathcal{X}: |\bigcup_{Y \in \mathcal{Y}} Y| > m$ tak pre 
    $\mathcal{X}$ existuje systém rozličných reprezentantov.
\end{theorem}
\begin{proof}
Cez Hallovu vetu pre bipartitné grafy. 
Vytvoríme bipartitný graf, ktorého partícia $A$ obsahuje jeden vrchol pre každú množinu 
z $\mathcal{X}$ a partícia $B$ obsahuje jeden vrchol pre každý prvok z $X$. Každý vrchol množiny spojíme s vrcholmi prvkov, 
ktoré obsahuje, teda $(X_i, x_j) \in E(G) \Leftrightarrow x_j \in X_i$. 
Z Hallovej vety o párení potom existuje párenie pokrývajúce všetky vrcholy $A$,
ak z každej podmnožiny vrcholov $A$ veľkosti $m$ vychádza aspoň $m$ hrán.  
\end{proof}

\begin{exercise}
\label{exer:findrepr}
Napíšte polynomiálny algoritmus na nájdenie systému rozličných reprezentantov.
\end{exercise}

\begin{theorem}
    Každý latinský obdĺžnik sa dá doplniť na latinský štvorec.
\end{theorem}
\begin{proof}
    Pomocou Hallovej vety dokážeme, že do každého latinského obdĺžnika $k \times n, k < n$ vieme pridať ďalší riadok.
    Pre $i$-ty stĺpec obdĺžnika definujeme množinu kandidátov $X_i$ ako prvky, ktoré sa v stĺpci nenachádzajú.
    Z latinskej vlastnosti vyplýva, že do každého stĺpca sa dá doplniť práve $n-k$ prvkov, 
    teda $\forall i \in \set{1, \ldots, n}: |X_i| = n-k$. Opačne, každý prvok sa dá doplniť do $n-k$ stĺpcov. 
    Bipartitný graf zodpovedajúci $\mathcal{X} = \set{X_1, \ldots X_n}$ je teda $(n-k)$-regulárny. 
    Z každej množiny stĺpcov veľkosti $m$ tak vychádza $m(n-k)$ hrán, ktoré musia v druhej partícii vchádzať do $m$ vrcholov. 
    Keďže je splnená Hallova podmienka, existuje systém reprezentantov $\mathcal{X}$, ktorý vieme použiť ako $(k+1)$-vý riadok
    latinského obdĺžnika.
\end{proof}

\begin{exercise}
Napíšte polynomiálny algoritmus na generovanie alebo doplňovanie latinských štvorcov (\emph{hint: použite algoritmus z úlohy \ref{exer:findrepr}}).
\end{exercise}

\section{Ortogonálne latinské štvorce}

\begin{definition}
\label{def:ols}
Nech $l = [\phi_1, \ldots, \phi_n]$ a $l' = [\psi_1, \ldots, \psi_n]$ sú latinské štvorce rádu $n$. 
Hovoríme, že $l$ a $l'$ sú ortogonálne (znáčime ako $l \perp l'$), ak platí: 
$$\forall i,j,k,l \in \set{1, \ldots, n}: (i, j) \neq (k, l) \Longrightarrow (\phi_i(j), \psi_i(j)) \neq  (\phi_k(l), \psi_k(l)).$$
Inak povedané, ak z prvkov na rovnakých pozíciách vytvoríme dvojice, tak sa každá dvojica z $\set{1, \ldots, n}^2$ objaví práve raz.
\end{definition}


\begin{exercise}
Nájdite dva ortogonálne latinské štvorce rádu 5.
\end{exercise}

\begin{exercise}
Napíšte (nie nutne polynomiálny) program, ktorý pre daný latinský štvorec nájde jemu ortogonálny.
\end{exercise}


\begin{theorem}
\label{thm:lsorto}
Nech $l = [\phi_1, \ldots, \phi_n]$ a $l' = [\psi_1, \ldots, \psi_n]$ sú latinské štvorce rádu $n$.
Zavedieme nasledovné značenia:
\begin{itemize}
    \item Nech $\lambda \in S_n$, potom $\lambda l := [\lambda \phi_1, \ldots, \lambda \phi_n]$ ($\lambda l$ je tiež latinský štvorec). 
    \item Nech kompozícia $l$ a $l'$ je definovaná ako $l \circ l' := [\phi_1 \psi_1, \ldots, \phi_n \psi_n]$.
\end{itemize}

Potom platí:

\begin{enumerate}
    \item $l \perp l' \Leftrightarrow [\psi_1 \phi_1^{-1}, \ldots, \psi_n \phi_n^{-1}]$ je latinský štvorec
    \item Ak $\lambda, \rho \in S_n$ a $l \perp l'$,  tak aj $\lambda l \perp \rho l'$
    \item $l \perp l' \Leftrightarrow$ existuje latinský štvorec $l''$ taký, že $l' = l'' \circ l$ 
\end{enumerate}
    
\end{theorem}


\begin{proof}
Prvé tvrdenie je ekvivalenciou, označíme ľavú stranu ako $A$ a pravú ako $B$. 
Dokazovať túto ekvivalenciu budeme postupne, obidve implikácie $A \Longrightarrow B$ a $B \Longrightarrow A$ samostatne.
Obidve implikácie budeme dokazovať nepriamo, t.j. dokážeme tvrdenia $\neg A \Longrightarrow \neg B$ a $\neg B \Longrightarrow \neg A$.
\begin{align*}
\neg A &\overset{(def.)}{\Longleftrightarrow} \exists i, j, k, l \in \set{1, \ldots, n}: (i, j) \neq (k, l) \wedge \phi_i(j) = \phi_k(l) \wedge \psi_i(j) = \psi_k(l)\\
\neg B &\overset{(def.)}{\Longleftrightarrow} \exists I, J, X \in \set{1, \ldots, n}: I \neq J \wedge \psi_I\phi_I^{-1}(X) = \psi_J \phi_J^{-1}(X)    
\end{align*}

\noindent$\neg A \Longrightarrow \neg B$:

Nech platí $\neg A$ a nech $i,j,k,l$ sú príslušné čísla z tvrdenia.
Nech\footnote{indexy $I$ a $J$ sme zvolili tak, lebo sú to jediné permutácie $\phi_i$, o ktorých niečo vieme. Podobne aj $X$, nakoľko vieme o špecifickom správaní daných permutácií len v tomto bode} $X := \phi_i(j) = \phi_k(l), I := i, J := k$.
Treba ukázať, že $I \overset{?}{\neq} J$ a $\psi_I\phi_I^{-1}(X) \overset{?}{=} \psi_J \phi_J^{-1}(X)$.
Začnime druhým:
\begin{align*}
\psi_I\phi_I^{-1}(X) &\overset{?}{=} \psi_J \phi_J^{-1}(X) &\text{dosadíme hodnoty $X, I, J$}\\
\psi_i \phi_i^{-1}(\phi_i(j)) &\overset{?}{=} \psi_k \phi_k^{-1}(\phi_k(l)) &\text{$\phi(\phi^{-1}(x)) = x$}\\
\psi_i(j) &\overset{\checkmark}{=} \psi_k(l) &\text{platí z predpokladu $\neg A$}
\end{align*}

Zostáva ešte ukázať, že $I \overset{?}{\neq} J$. Sporom: nech $I = J$, čiže $i = k$. 
Potom nutne $j \neq l$ (z predpokladu $\neg A$).
Po dosadení do tvrdenia $\neg A$ dostávame $j \neq l \wedge \phi_i(j) = \phi_i(l)$, t.j. permutácia $\phi_i$ zobrazuje dva rôzne prvky $j$ a $l$ do toho istého prvku, čo je spor s tým, že $\phi_i$ je injektívne zobrazenie.
Týmto je dôkaz prvej implikácie z prvého tvrdenia vety ukončený.

\noindent$\neg B \Longrightarrow \neg A$:

Nech platí $\neg B$ a $I, J, X$ sú príslušné čísla z tvrdenia.
Nech $i := I$, $k := J$, $j := \phi_I^{-1}(X)$, $l := \phi_J^{-1}(X)$.
Treba ukázať, že $(i, j) \overset{?}{\neq} (k, l)$, $\phi_i(j) \overset{?}{=} \phi_k(l)$ a $\psi_i(j) \overset{?}{=} \psi_k(l)$.

Prvé tvrdenie je platné kvôli predpokladu $I \neq J \Longleftrightarrow i \neq k$.

Druhé tvrdenie:
\begin{align*}
\phi_i(j) &\overset{?}{=} \phi_k(l)\\
\phi_I(\phi_I^{-1}(X)) &\overset{?}{=} \phi_J(\phi_J^{-1}(X))\\
X &\overset{\checkmark}{=} X\\
\end{align*}

Tretie tvrdenie:
\begin{align*}
\psi_i(j) &\overset{?}{=} \psi_k(l)\\
\psi_I(\phi_I^{-1}(X)) &\overset{\checkmark}{=} \psi_J(\phi_J^{-1}(X))&\text{z predpokladu $\neg B$}
\end{align*}

Týmto je dôkaz druhej implikácie, a s tým aj prvého tvrdenia z vety celkovo ukončený.
Dôkaz druhého a tretieho tvrdenia prenechávame čitateľovi ako samostatné cvičenie (úlohy \ref{ex:lsorto1} a \ref{ex:lsorto2}).
\end{proof}

\begin{exercise}
\label{ex:lsorto1}
Dokážte tvrdenie 2 z vety \ref{thm:lsorto}.
\end{exercise}

\begin{exercise}
\label{ex:lsorto2}
Dokážte tvrdenie 3 z vety \ref{thm:lsorto}.
\end{exercise}




\begin{definition}
Množinu vzájomne ortogonálnych latinských štvorcov budeme označovať $MOLS$ a množinu vzájomne ortogonálnych latinských štvorcov rádu $n$ označíme $MOLS(n)$.
\end{definition}

\begin{theorem}
Maximálna (vzhľadom na inklúziu) $MOLS(n)$ má najviac $n-1$ prvkov.
\end{theorem}


\begin{proof}
Sporom: nech dokazované tvrdenie neplatí, teda existuje $n$ navzájom ortogonálnych latinských štvorcov $l_1, \ldots, l_n$ rádu $n$.
Preusporiadajme stĺpce latinských štvorcov tak, aby prvý riadok v každom bol rovný $(1, \ldots, n)$ (z vety \ref{thm:lsorto}).
Pozrime sa teraz na prvé políčko v druhom riadku všetkých $n$ latinských štvorcov.

Tieto políčka nesmú obsahovať $1$, lebo v prvom stĺpci je už použitá.
Zároveň však platí, že hodnoty v týchto políčkach musia byť navzájom rôzne, nakoľko sa páry typu $(k, k)$ už vyskytujú pri prvom riadku v každom páre latinských štvorcov (z definície ortogonálnych latinských štvorcov, každá dvojica hodnôt, vytvorená z prvkov na rovnakých pozíciách, sa musí vyskytnúť práve raz).
Nakoľko vybrať z $(n-1)$ hodnôt až $n$ rôznych nie je možné, dostávame spor, čím je dôkaz ukončený.
\end{proof}


\begin{definition}
Latinský štvorec je v normálnom tvare, ak prvý riadok tabuľky je rovný $(1, \ldots, n)$ a prvý stĺpec je rovný $(1, \ldots, n)^T$.
\end{definition}

\begin{definition}
Latinské štvorce $l$ a $l'$ sú izotopické, ak sa dajú permutáciou riadkov, stĺpcov a názvov prvkov previesť na rovnaký latinský štvorec v normálnom tvare.
\end{definition}

\begin{remark}
Latinský štvorec v normálnom tvare zodpovedá tabuľke binárnej operácie kvazigrupy 
(kvazigrupa je množina $Q$ s binárnou operáciou $\circ$, v ktorej pre každý prvok $a, b \in Q$ existuje práve jedno $x, y \in Q$ také, že platia rovnosti $a \circ x = b$ a $y \circ a = b$)\footnote{dá sa to neformálne predstaviť ako klasickú grupu bez zaručenej asociativity}.

Platí, že 2 kvazigrupy sú izomorfné práve vtedy, keď príslušné latinské štvorce sú izotopické.
\end{remark}


\begin{exercise}
Nájdite dva neizotopické latinské štvorce rádu $5$.
\end{exercise}


\begin{exercise}
Napíšte program, ktorý pre dva zadané latinské štvorce overí, či sú izotopické.
\end{exercise}



\begin{definition}
Latinský štvorec si vieme predstaviť ako maximálnu (na inklúziu) množinu $A$ trojíc $(r,c,s) \in \set{1, \ldots, n}^3$, 
kde $r$ zodpovedá číslu riadku, $c$ zodpovedá číslu stĺpca a $s$ zodpovedá hodnote v políčku $(i, j)$, 
takú, že platí:
\begin{itemize}
    \item všetky dvojice $(r, c)$ sú rôzne (''máme $n^2$ políčok'')
    \item všetky dvojice $(r, s)$ sú rôzne (''v každom riadku sa vyskytnú všetky hodnoty od $1$ po $n$'')
    \item všetky dvojice $(c, s)$ sú rôzne (''v každom stĺpci sa vyskytnú všetky hodnoty od $1$ po $n$'')
\end{itemize}  

Konjugáciou latinského štvorca voláme množinu trojíc $A'$, ktorá vznikne z $A$ permutáciou trojíc. Formálne,
nech $\lambda \in S_3$ je permutácia veľkosti $3$, potom $$A' = \set{ (a_{\lambda(1)}, a_{\lambda(2)}, a_{\lambda(3)} ) ~|~ (a_1, a_2, a_3) \in A}$$

Latinské štvorce, ktoré sa dajú jeden z druhého dostať pomocou konjugácie, voláme \emph{konjugované}.
Latinské štvorce, ktoré sa dajú jeden z druhého dostať pomocou konjugácie a izotopie, voláme \emph{paratopické}.

\end{definition}



\begin{exercise}
Nájdite dva neparatopické latinské štvorce rádu 5.
\end{exercise}

\begin{exercise}
Napíšte program, ktorý pre dva zadané latinské štvorce rozhodne, či sú paratopické.
\end{exercise}


\begin{theorem}{(Stevens, 1935)}
\label{thm:stevens}
Ak $n = p^\alpha$, kde $p$ je prvočíslo, tak maximálna $MOLS(n)$ má $n-1$ prvkov.
\end{theorem}

\begin{construction}
Číslo $n$ je mocninou prvočísla, preto existuje konečné pole\footnote{nie všetci vedia, ale konečné polia sa volajú tak kvôli tradičnému pokriku študentov informatiky ''Konečne sú na niečo dobre!'', ktorý im samovoľne vyletí z úst, keď prvýkrát zbadajú tieto štruktúry mimo druhého semestra vyššej algebry} $F := GF(n)$ príslušnej veľkosti. 
Očíslujeme prvky poľa $F$ v ľubovoľnom poradí, ale nech $a_0 = 0$.

$k$-tý latinský štvorec si označme ako $l_k = \left(a_{ij}^{(k)}\right)$.

$a_{ij}^{(k)} := a_i a_k + a_j$
\end{construction}


\begin{proof}
Dôkaz pozostáva z dvoch častí.
Najprv treba ukázať, že skonštruované $l_k$ sú naozaj latinskými štvorcami, následne treba ukázať, že každá dvojica $(l_a, l_b)$ je ortogonálna.

Latinský štvorec musí spĺňať podľa definície dve podmienky:
\begin{enumerate}
    \item v každom riadku je práve raz každé z čísel $\set{1, \ldots, n}$, respektíve $$\forall i, j_1, j_2 \in \set{1, \ldots, n}: j_1 \neq j_2 \Longrightarrow a_{ij_1}^{(k)} \neq a_{ij_2}^{(k)},$$
    \item v každom stĺpci je práve raz každé z čísel $\set{1, \ldots, n}$, respektíve $$\forall i_1, i_2, j \in \set{1, \ldots, n}: i_1 \neq i_2 \Longrightarrow a_{i_1j}^{(k)} \neq a_{i_2j}^{(k)}.$$
\end{enumerate}

Sporom, nech prvá podmienka neplatí, čiže $$\exists i, j_1, j_2  \in \set{1, \ldots, n}: j_1 \neq j_2 \wedge a_{ij_1}^{(k)} = a_{ij_2}^{(k)}.$$

Pozrieme sa bližšie na výraz $a_{ij_1}^{(k)} = a_{ij_2}^{(k)}$:
\begin{align*}
a_{ij_1}^{(k)} &= a_{ij_2}^{(k)}\\
a_i a_{k} + a_{j_1} &= a_i a_{k} + a_{j_2}\\
a_{j_1} &= a_{j_2}&\text{SPOR}
\end{align*}

Dôkaz druhej podmienky sporom privedie ku výrazu $a_{i_1}a_k + a_j = a_{i_2}a_k + a_j$, ktorý taktiež vedie ku sporu.

Teraz zostáva dokázať $\forall k_1, k_2 \in \set{1, \ldots, n}: k_1 \neq k_2 \Longrightarrow l_{k_1} \perp l_{k_2}$.
Sporom, nech $\exists k_1, k_2: k_1 \neq k_2 \wedge l_{k_1} \not\perp l_{k_2}$, čiže $$\exists i, j, k, l: (i, j) \neq (k, l) \wedge a^{(k_1)}_{ij} = a^{(k_1)}_{kl} \wedge a^{(k_2)}_{ij} = a^{(k_2)}_{kl}.$$
Prepíšeme tento výrok podľa konštrukcie a dostaneme:
$$
(i, j) \neq (k, l) \wedge a_i a_{k_1} + a_j = a_k a_{k_1} + a_l \wedge a_i a_{k_2} + a_j = a_k a_{k_2} + a_l.
$$
Po zopár algebraických úpravách dostaneme:
$$
(i, j) \neq (k, l) \wedge (a_i - a_k) a_{k_1} = a_l - a_j = (a_i - a_k)a_{k_2}.
$$

Ak $i \neq k$, tak $a_i - a_k \neq 0$, čiže $a_{k_1} = a_{k_2} \Longrightarrow k_1 = k_2$, čo je spor.

Ak $j \neq l$, tak $a_l - a_j \neq 0$, čiže $(a_i - a_k) a_{k_1} \neq 0 \Longrightarrow a_i \neq a_j \Longrightarrow a_{k_1} = a_{k_2} \Longrightarrow k_1 = k_2$, čo je spor.

Týmto je dôkaz správnosti konštrukcie ukončený.

\end{proof}

\begin{example}
Povedzme, že chceme zostrojiť tri navzájom ortogonálne latinské štvorce. Budeme postupovať podľa konštrukcie z predchádzajúcej vety \ref{thm:stevens}. Najprv teda potrebujeme zostrojiť konečné pole veľkosti $4$. To sa dá napríklad faktorizáciou okruhu $\mathbb{Z}_2[x]$ podľa polynómu $x^2 + x + 1$, ktorý je ireducibilný nad $\mathbb{Z}_2$. Prvky $GF_4 = \mathbb{Z}_2[x]/(x^2 + x + 1)$ sú $\set{0, 1, x, x+1}$. Zostrojme multiplikačné tabuľky pre toto pole:
\begin{center}
    \begin{tabular}{|c||c c c c|}
        \hline
        $+$ & $0$ & $1$ & $x$ & $x + 1$ \\ 
        \hline
        \hline
        $0$ & $0$ & $1$ & $x$ & $x + 1$ \\ 
        $1$ & $1$ & $0$ & $x + 1$ & $x$ \\
        $x$ & $x$ & $x + 1$ & $0$ & $1$ \\ 
        $x + 1$ & $x + 1$ & $x$ & $1$ & $0$ \\
        \hline
    \end{tabular}
    \begin{tabular}{|c||c c c|}
        \hline
        $*$ & $1$ & $x$ & $x + 1$ \\ 
        \hline
        \hline
        $1$ & $1$ & $x$ & $x + 1$ \\
        $x$ & $x$ & $x + 1$ & $1$ \\ 
        $x + 1$ & $x + 1$ & $1$ & $x$ \\
        \hline
    \end{tabular}
\end{center}
Očíslujeme prvky poľa $GF_4$ ako $a_0 = 0, a_1 = 1, a_2 = x, a_3 = x+1$. Teraz už máme všetko potrebné aby sme zostrojili požadované latinské štvorce. Prvý latinský štvorec bude mať prvky $a_{ij}^{(1)} = a_i a_1 + a_j$, pre $i, j \in \set{0, 1, 2, 3}$:
$$
\begin{pmatrix}
    0 & 1 & 2 & 3\\
    1 & 0 & 3 & 2\\
    2 & 3 & 0 & 1\\
    3 & 2 & 1 & 0
\end{pmatrix}
$$
Napríklad prvok štvorca v prvom riadku a druhom stĺpci (indexujúc od nuly) je $a_{12}^{(1)} = a_1 a_1 + a_2 = 1 \cdot x + 1 = x + 1 = a_3$. Druhý a tretí štvorec zostrojíme analogicky:
$$
\begin{pmatrix}
    0 & 1 & 2 & 3\\
    2 & 3 & 0 & 1\\
    3 & 2 & 1 & 0\\
    1 & 0 & 3 & 2
\end{pmatrix}
\begin{pmatrix}
    0 & 1 & 2 & 3\\
    3 & 2 & 1 & 0\\
    1 & 0 & 3 & 2\\
    2 & 3 & 0 & 1\\
\end{pmatrix}
$$
\end{example}

\begin{definition}
Množina $n - 1$ vzájomne ortogonálnych latinských štvorcov rádu $n$ sa nazýva úplná $MOLS(n)$.
\end{definition}

\begin{exercise}
Napíšte program, ktorý pre zadané prvočíslo $p$ zostrojí úplnú $MOLS(n)$.
\end{exercise}

\begin{exercise}
Napíšte program, ktorý pre zadané prvočíslo $p$ a číslo $\alpha$ zostrojí úplnú $MOLS(p^\alpha)$.
\end{exercise}

\begin{theorem_hard}{(Bose, Parker, Schrickhande, 1960)}

$\forall n \geq 3 \wedge n \neq 6:$ existujú aspoň dva vzájomne ortogonálne latinské štvorce rádu $n$.

\end{theorem_hard}

\begin{remark}
Táto veta, má za sebou zaujímavý historický príbeh. Euler sa zaoberal problémom zostrojovania ortogonálnych latinských štvorcov ešte v 18. storočí. Katarína Veľká ho požiadala, aby navrhol ako usporiadať $36$ dôstojníkov s $6$ hodnosťami zo $6$ plukov do štvorca tak, aby v každom riadku a každom stĺpci boli dôstojníci rôznej hodnosti a aj z rôznych plukov. Inými slovami aby zostrojil dva ortogonálne latinské štvorce rádu $6$, v jednom by boli usporiadané hodnosti a v druhom pluky dôstojníkov. Ich prekrytie by tvorilo požadované usporiadanie dôstojníkov.

Eulerovi sa tento problém nepodarilo vyriešiť, ale počas svojho bádania navrhol postup pre zostrojovanie dvoch vzájomne ortogonálnych latinských štvorcov rádu $n$ pre nepárne $n$ a $n$, ktoré je násobkom $4$. Popritom odpozoroval, že nie je možné zostrojiť dva vzájomne ortogonálne latinské štvorce rádu $2$ a keďže nebol schopný zostrojiť štvorce rádu $6$, tak vyslovil hypotézu, že pre nepárne párne\footnote{Viac o nepárne párnych číslach sa dá dočítať na \url{ https://en.wikipedia.org/wiki/Singly_and_doubly_even}} $n \equiv 2\ (\textrm{mod}\ 4)$ neexistujú vzájomne ortogonálne latinské štvorce rádu $n$.

V roku 1959 sa však podarilo zostrojiť dva vzájomne ortogonálne latinské štvorce rádu $10$ pomocou hrubej sily a počítača, čo bolo jedno z prvých použití výpočtovej sily na riešenie matematických problémov. Neskôr v tom roku vydali Bose, Parker a Schrickhande článok, v ktorom dokázali, že vzájomne ortogonálne latinské štvorce existujú pre každé $n > 1$ okrem $n = 2$ a $n = 6$. Týmto definitívne vyvrátili Eulerovu hypotézu.

Euler teda spozoroval výnimku na dvoch prípadoch a mylne ju generalizoval na všetky nepárne párne rády štvorcov. No vyvrátiť jeho hypotézu trvalo približne $200$ rokov.    
\end{remark}


\begin{definition}
Nech $A \in \mathbb{R}^{n\times m}$ a $B \in \mathbb{R}^{p \times q}$ sú matice ľubovoľnej veľkosti.
Potom matica $A \otimes B \in \mathbb{R}^{np \times mq}$ je Kronekerovym súčinom matíc $A$ a $B$, ak platí:
$$
A \otimes B = 
\begin{bmatrix} a_{1 1} B & \cdots & a_{1 m} B \\ \vdots & \ddots & \vdots \\ a_{n 1} B & \cdots & a_{n m} B
\end{bmatrix}
$$
\end{definition}

\begin{example}
Kroneckerov súčin môže byť napríklad:
$$
\begin{bmatrix}
1 & 2 & 3 \\ 4 & 5 & 6
\end{bmatrix}
\otimes
\begin{bmatrix} 1 & 1 \\ -1 & 2 \end{bmatrix}
= 
\begin{bmatrix}
1 \begin{bmatrix} 1 & 1 \\ -1 & 2 \end{bmatrix}  &  2 \begin{bmatrix} 1 & 1 \\ -1 & 2 \end{bmatrix} & 3 \begin{bmatrix} 1 & 1 \\ -1 & 2 \end{bmatrix} \\ \\
4 \begin{bmatrix} 1 & 1 \\ -1 & 2 \end{bmatrix} & 5 \begin{bmatrix} 1 & 1 \\ -1 & 2 \end{bmatrix} & 6\begin{bmatrix} 1 & 1 \\ -1 & 2 \end{bmatrix}
\end{bmatrix}
=
\begin{bmatrix}
1 & 1 & 2 & 2 & 3 & 3\\
-1 & 2 & -2 & 4 & -3 & 6\\
4 & 4 & 5 & 5 & 6 & 6\\
-4 & 8 & -5 & 10 & -6 & 12
\end{bmatrix}
$$
\end{example}

\begin{theorem_hard}{(Vlastnosti Kronekerovho súčinu)}
\label{th:kr_formulas}
\begin{enumerate}
    \item $(A \otimes B)^T = A^T \otimes B^T$
    \item $(A + B) \otimes C = A \otimes C + B \otimes C$
    \item $(A B) \otimes (C D) = (A \otimes C)(B \otimes D)$
\end{enumerate}
\end{theorem_hard}

\begin{theorem}
\label{thm:prime_ols}
$|MOLS(n_1)| \geq m \wedge |MOLS(n_2)| \geq m \Rightarrow |MOLS(n_1 n_2)| \geq m$
\end{theorem}

\begin{construction}
$k$-tý latinský štvorec rádu $n_1 n_2$ sa dá získať pomocou Kroneckerovho súčinu $k$-tých príslušných latinských štvorcov rádu $n_1$ a $n_2$.

Formálne, nech $l_1, \ldots, l_m$ sú ortogonálne latinské štvorce rádu $n_1$ a $l'_1, \ldots, l'_m$ sú ortogonálne latinské štvorce rádu $n_2$.
Potom množina matíc $\set{l_k \otimes l'_k ~|~ k \leq m}$, kde $\otimes$ je Kroneckerov súčin matíc, je množina ortogonálnych latinských štvorcov rádu $n_1 n_2$.

Overenie správnosti tejto konštrukcie prenechávame čitateľovi ako samostatné cvičenie (úloha \ref{ex: prime_ols}).
\end{construction}

\begin{corollary}
$$n = p_1^{\alpha_1} p_2^{\alpha_2} \ldots p_r^{\alpha_r} \Rightarrow |MOLS(n)| \geq \min_{i \leq r} (p_i^{\alpha_i} - 1)$$
\end{corollary}


\begin{exercise}
\label{ex: prime_ols}
Dokážte správnosť konštrukcie z vety \ref{thm:prime_ols}.
\end{exercise}


\begin{theorem}
\label{thm:odd_ols}
$$n = 2m - 1 \Rightarrow |MOLS(n)| \geq 2$$
\end{theorem}

\begin{construction}

Pohybujeme sa v cyklickej grupe $(\mathbb{Z}_n, +) = \set{0, \ldots, n-1}$.

\begin{align*}
A := (a_{ij}),~& a_{ij} := m (i+j)~(mod~n) \\
B := (b_{ij}),~& b_{ij} := (i-j)  ~(mod~n)    
\end{align*}

Overenie správnosti tejto konštrukcie prenechávame čitateľovi ako samostatné cvičenie (úloha \ref{ex:odd_ols}).
\end{construction}


\begin{exercise}
\label{ex:odd_ols}
Dokážte správnosť konštrukcie z vety \ref{thm:odd_ols}.
\end{exercise}


\section{Traverzály}
Traverzála latinského štvorca je taký výber $n$ políčok, že v každom riadku je vybraté práve jedno políčko, v každom stĺpci je vybraté práve jedno políčko a každý symbol je obsiahnutý v práve jednom políčku.

\begin{definition}
Nech $A := (a_{ij})$ je latinský štvorec veľkosti $n$. $n$-tica $\{(i_1, j_1), (i_2,j_2), \dots , (i_n,j_n)\}$ je traverzála latinského štvorca $A$ p. v. k.  $\{a_{i_1j_1},a_{i_2j_2}, \dots, a_{i_nj_n}\} = \{i_1, i_2, \dots, i_n \} = \{ j_1, j_2, \dots, j_n\} = \{1, 2, \dots, n \}$
\end{definition}

\begin{exercise}
Nájdite jeden latinský štvorec, ktorý má traverzálu a jeden, ktorý nemá traverzálu. 
\end{exercise}

\begin{exercise}
Nech $A := (a_{ij}),a_{ij} = i+j $ je latinský štvorec veľkosti $n$. Ukážte, že $A$ nemá traverzálu.
\end{exercise}

\begin{theorem}
Nech $A := (a_{ij}), B:=(b_{ij})$ sú kolmé latinské štvorce veľkosti $n$, $c \in \{1,2,\dots,n\} , T = \{ (i,j) | a_{ij} = c\}$. Potom $T$ je traverzála štvorca $B$.
\end{theorem}

\begin{proof}
Keďže $A$ je latinský štvorec, tak platí, že veľkosť $T$ je $n$ a pre každé dva rôzne prvky $(i_1, j_1), (i_2,j_2) \in T$ platí $i_1 \neq i_2 ,j_1 \neq j_2$ 
z čoho vyplýva $\{i_1,i_2, \dots, i_n\} = \{j_1, j_2, \dots, j_n\} = \{ 1, 2, \dots, n\}$.

Keďže $A$ a $B$ sú na seba kolmé, tak $b_{i_k j_k}$ a $b_{i_l j_l} ((i_k, j_k), (i_l, j_l) \in T, (i_k, j_k) \neq (i_l, j_l))$ musia byť rôzne a teda $\{b_{i_1j_1},b_{i_2j_2}, \dots b_{i_nj_n}\} = \{1,2, \dots , n\}$.
\end{proof}

\begin{corollary}
Latinský štvorec veľkosti $n$ má kolmú dvojičku, práve vtedy keď sa dá pokryť $n$ traverzálami.
\end{corollary}


